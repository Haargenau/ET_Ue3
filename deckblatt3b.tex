% -*- TeX:de -*-
\NeedsTeXFormat{LaTeX2e}
\documentclass[12pt,a4paper,titlepage]{article}

%\usepackage[german]{babel} % german text
\usepackage[DIV12]{typearea} % size of printable area
\usepackage[T1]{fontenc} % font encoding
\usepackage[utf8]{inputenc} % probably on Linux

\usepackage{graphicx} % to include images
\usepackage{subfigure} % for creating subfigures
\usepackage{amsmath} % a bunch of symbols
\usepackage{amssymb} % even more symbols
\usepackage{booktabs} % pretty tables
\usepackage{csquotes}

% a floating environment for circuits
\usepackage{float}
\usepackage{caption}

\usepackage[european]{circuitikz}
\ctikzset{voltage/distance from line=.25}% pos. between 0 and 1

\newfloat{circuit}{tbph}{circuits}
\floatname{circuit}{Schaltplan}

% a floating environment for diagrams
\newfloat{diagram}{tbph}{diagrams}
\floatname{diagram}{Diagramm}

\renewcommand{\familydefault}{\sfdefault} % activate to use sans-serif font as default

\sloppy % friendly typesetting

\usepackage{eurosym}
%\usepackage{makeidx}
\usepackage{amsfonts}
\usepackage{mparhack}
\usepackage{array}
\usepackage{tabularx}
\usepackage{minitoc}
\usepackage[colorlinks=true]{hyperref}
\usepackage{epstopdf}
\usepackage{setspace}
\usepackage{csquotes}

\usepackage{pgfplots}
\usetikzlibrary{automata,arrows,chains,shapes.misc,scopes,petri}

\usepackage{csvsimple}
\usepackage{siunitx,array,booktabs}
\usepackage{longtable}
\usepackage[pass]{geometry}


\ctikzset{voltage/distance from node=.5}% in \pgf@circ@Rlen units
\ctikzset{voltage/distance from line=.33}% pos. between 0 and 1
\ctikzset{voltage/bump b/.initial=1.5}%


\begin{document}

\begin{titlepage}

\begin{figure*}[h!]
  \includegraphics[width=8cm]{TULogo_CMYK}
\end{figure*}

\begin{center}
\vspace*{1.3cm}
{\Huge Elektrotechnische Grundlagen der Informatik\\(LU 182.692)\\}
\vspace{1.7cm}
{\LARGE Protokoll der 3. Laborübung: \enquote{Operationsverstärker}\\}
{\LARGE b) Messungen\\}
\vspace{1.5cm}

% fill in group number and date of lab here
% CHANGE ME!
{\Large Gruppennr.: 10 \hspace{1cm} Datum der Laborübung: 01.06.2017}

% fill in IDs and names here
% CHANGE ME!
\begin{table}[h!]
\centering
\begin{tabular}{|p{3.5cm}|p{3.5cm}|p{6.5cm}|}
\hline \textbf{Matr. Nr.} & \textbf{Kennzahl} & \textbf{Name} \\
\hline
1609418 & 033 535 & GEISELBRECHTINGER Max \\
\hline
1625753 & 033 535 & HAAR Martin \\
\hline
& & \\
\hline
\end{tabular}
\end{table}

\end{center}
\vspace{1.0cm}

\begin{table}[h!]
\begin{tabular}{|l|l|}
\hline \textbf{Kontrolle} & \checkmark \\
\hline Nichtinvertierender OPV & \\
\hline OPV und Grenzfrequenz & \\
\hline Invertierender OPV & \\
\hline Integrierer & \\
\hline Schmitt-Trigger & \\
\hline
\end{tabular}
\end{table}

\end{titlepage}
\setcounter{page}{2}

%only sections in tableofcontents, no subsections
\setcounter{tocdepth}{1}
%sets tableofcontents color black
\hypersetup{linkcolor=black}

% start of actual lab protocol
% CHANGE ME!
\tableofcontents
% !TEX root = deckblatt3b.tex

\section{Nicht-Invertierender Verst\"arker}

% !TEX root = deckblatt3b.tex

\section{Invertierender Verst\"arker}

\begin{figure}[H]
  \begin{center}
    \begin{circuitikz}[scale=1.3]
      \draw
    (0,0) to[sinusoidal voltage source,v<=$U_1$] (0,2) % The voltage source
          to[L=$1mH$] (3,2)
          to[R=$22\Omega$] (6,2)
          to[C=$100nF$] (6,0)
          to[short] (0,0);
    \end{circuitikz}
    \caption{RLC-Glied Messschaltung.}
  \end{center}
\end{figure}

% !TEX root = deckblatt3b.tex

\section{Integrierer}
\subsection{Aufgabenstellung}
Es soll ein Integrierer aufgebaut und dessen Funktion überpr\"uft werden. Anschlie\ss{}end soll die Funktion des Integrierers erl\"autert werden.
\subsection{Schaltung}
\begin{figure}[H]
  \begin{center}
    %\tikzset{component/.style={draw,thick,circle,fill=white,minimum size =0.75cm,inner sep=0pt}}
      \begin{circuitikz}
      \draw
      (-2,2) node[ground] (ground) {}
      (3,4) node[op amp] (opamp) {}
      (3,4.5) to[short] (3,5) node[vcc] {$V_{CC}$}
      (3,3.5) to[short] (3,3) node[vss] {$V_{SS}$}
      (opamp.-) to[R={$R$}{$=2,2k$},v=$U_R$] (-2,4.5)
      (opamp.+) to[short] (1.5,3.5) to[short] (1.5,2) node[ground] (ground) {}

      (1.5,4.5) to[short] (1.5,7.5) to[C={$C$}{$=1\mu$},v=$U_C$] (4.5,7.5) to[short] (4.5,4)
      (1.5,7.5) to[short] (1.5,9) to[R=$220k$] (4.5,9) to[short] (4.5,7.5)
      (opamp.out) to[short] (6,4)
      (6,2) node[ground] (ground) {}
      ;
      \draw (-2,4.2) to[short] (-2,2.7) node[vee] {};
      \draw (-1.5,3) node[] {$U_e$};
      \draw (6,3.7) to[short] (6,2.7) node[vee] {};
      \draw (6.5,3) node[] {$U_a$};
      \draw (4.1,3.5) node[] {LM741};
      \end{circuitikz}
    \caption{nicht inververtierender OPV}
  \end{center}
\end{figure}
\noindent
F\"ur die Schaltung des Integrierers wird die Grundschatung des invertierenden Verst\"arkers verwendet, es wird lediglich der R\"uckkopplungs Widerstand durch einen Kondensator ersetzt. Der Widerstand parallel zum Kondensator wird um ein Vielfaches gr\"o\ss{}er gew\"ahlt als der Widerstand am Eingang ($100*R_1$ in Abb. 15). Dieser beeinflusst die Schaltungseigenschaften nicht, er dient der Entladung des Kondensator vor dem Einschaltzeitpunkt um die Anfangsbedingung ($U_a=0V$) des Integrators zu bewerkstelligen.
\newpage
\subsection{Messung im Zeitbereich}
F\"ur die Messung im Zeitbereich wurde am Eingang ein Rechtecksignal mit $f=5Hz$ und einer Amplitude von $0.1Vpp$ angelegt.

\begin{figure}[H]
 \begin{center}
  \includegraphics[height=6cm,width=12cm]{OsziBilder/invInte_bigScal.png}
 \end{center}
 \caption{Rechteckspannung $0,1V_{pp}$, $5Hz$ ($U_e$ orange, $U_a$ grün)}
\end{figure}
\noindent
Jede Halbwelle des Eingangssignales entspricht einem Sprung und dieser integriert ergibt eine Gerade. Da es sich um einen invertierenden Integrierer handelt l\"auft die Gerade immer entegen der Sprungrichtung. In der Abbildung ist genau dieses Verhalten zu erkennen, springt die Eingangspannung ins Negative, entspricht die Ausgnagsspannung einer Geraden mit positiver Steigung, springt die Eingangsspannung ins Positive so ist die Ausgangsspannung eine Gerade mit negativer Steigung.
Die Ausgangsspannung ist, wie man in der Abbildung sehen kann, keine genaue gerade, sondern der Anfang der Lade- bzw. Entladekurve des Kondensators. Wird jedoch über ein kleines Intervall integriert, so entspricht dies annähernd einer Geraden.\\
\newpage
\subsection{Erl\"auterung der Schaltung}
\begin{figure}[H]
  \centering
  \begin{tabular}{ccc}
    & & Ohm'sches Gesetz an $C$ \\
    & & $I=C*\frac{du}{dt}$ \\ \\
    Ohm'sches Gesetz an $R$& & $du = \frac{I*dt}{C}$ \\ \\
    $I=\frac{U_e}{R}$ & & $U_a=\frac{1}{C}\int I dt$ \\
    & Zusammenf\"uhrung der Formeln & \\
    & $U_a = -\frac{1}{C} \int \frac{U_e}{R} dt$ & \\ \\
    & $U_a = -\frac{1}{RC} \int U_e dt$ & \\
  \end{tabular}
  \caption{Formeln zur Berechnung des Integrierers}
\end{figure}
\noindent
Da auf Grund des hohen Eingangswiderstandes kein Strom in den OPV flie\ss{}t, ist auch der Strom durch den Widerstand und Kondensator nahezu ident. Dieser Strom l\"asst sich sowohl am Widerstand als auch am Kondensator durch das Ohm'sche Gesetz berechnen. Formt man nun das Ohm'sche Gesetz von $C$ nach $du$ um, so erh\"alt man eine Differntialgleichung 1. Ordnung. Diese l\"asst sich durch Integrieren l\"osen und es ergibt sich eine Formel zur Berechnung von $U_a$. Setzt man nun die Berechnung f\"ur $I$ am Widerstand in die Umgeformte Formel von $C$ ein erh\"alt man die Formel zur Berechnung der Ausgangsspannung. \\ \\
$U_a = -\frac{1}{RC} \int U_e dt$ \\  \\
Diese Formel besagt, dass die Ausgangsspannung das Integral der Eingangsspannung mit dem Proportionalit\"atsfaktor $\frac{1}{RC}$ ist. \\ \\
Die Integriererschaltung wird häufig in der Regelungstechnik, zB. f\"ur Integralrelger (PI-, PID-Regler), verwendet.

\newpage

% !TEX root = deckblatt3b.tex

\section{Invertierender Schmitt-Trigger}

In dieser Aufgabe war der Operationsverstärker als invertierender Schmitt-Trigger zu beschalten und dessen Funktionsweise zu erproben.\\

\subsection{Schaltung}

\begin{figure}[H]
  \begin{center}
    %\tikzset{component/.style={draw,thick,circle,fill=white,minimum size =0.75cm,inner sep=0pt}}
    \resizebox{0.7\linewidth}{!}{
      \begin{circuitikz}
      \draw (0,0)
      (3,4) node[op amp] (opamp) {}
      (3,4.5) to[short] (3,5) node[vcc] {$V_{CC}$}
      (3,3.5) to[short] (3,3) node[ground](ground) {}
      (opamp.-) to[short] (-2,4.5)
      (opamp.+) to[short] (1.5,3.5) to[short] (1.5,1.5) to[short] (4.5,1.5){}
      (4.5,4) to[R={$R_2$}] (4.5,1.5) to[R={$R_1$}] (4.5,-0.5) node[ground] (ground) {}
      (opamp.out) to[short] (8,4)
      (1.5,1.5) to[R={$R_3$}] (-1, 1.5) to [voltage source, v>=$V_{CC}$] (-1, -0.5) node[ground](ground) {}
      (8,-0.5) node[ground] (ground) {}
      (-2,-0.5) node[ground] (ground) {}
      ;
      \draw (-2,4.2) to[short] (-2,0.2) node[vee] {};
      \draw (-2.5,2.2) node[] {$U_e$};
      \draw (8,3.7) to[short] (8,0.2) node[vee] {};
      \draw (8.5,1.7) node[] {$U_a$};
      \draw (4,1.1) to[short] (4,0.3) node [vee] {};
      \draw (3.5,0.6) node[] {$U_t$};

      \end{circuitikz}
    }  
    \caption{inververtierender Schmitt-Trigger}
  \end{center}
\end{figure}
\noindent

\begin{figure}[H]
 \begin{center}
  \includegraphics[height=6cm,width=12cm]{OsziBilder/SchmittTrigger_Time}
 \end{center}
 \caption{Zeitverlauf invertierender Schmitt-Trigger, $U_e$ orange und $U_a$ grün}
\end{figure}
\noindent
Der Schmitt-Trigger ist eine mitgekoppelte Operationsverstärkerschaltung. Seine Schaltschwellen liegen im allgemeinen symmetrisch
um den Nullpunkt. Durch das Anlegen einer Referenzspannung können die Schaltpunkte für das Über- bzw Untersteuern auch nicht
symmetrisch angesetzt werden. Wird die positive Schaltschwelle erreicht, wird am Ausgang die negative Versorgungsspannung
des OPV's ausgegeben (hier $0V$, Masse). Wird die negative Schaltschwelle erreicht, wird die positive Versorgungsspannung
ausgegeben (hier $V_{CC} = 5V$), diese wird allerdings nicht ganz erreicht, da es sich hier nicht um eine Rail-to-Rail OPV handelt.\\
\\
$U_e=$ Sinus $5V_{pp}$ $50Hz$ Offset $2,5V$, $V_{CC} = +5V$, $R_1=R_2=4,7k\Omega$, $R_3=10k\Omega$\\
gemessen: $U_a=3,18V$, $U_t=1,72V$\\

\subsection{Hysterese}

\begin{figure}[H]
 \begin{center}
  \includegraphics[height=6cm,width=12cm]{OsziBilder/SchmittTrigger_XY}
 \end{center}
 \caption{Hysterese-Kennlinie, x-Achse $U_e$, y-Achse $U_a$}
\end{figure}
\noindent
Die Hysterese-Kennlinie zeigt die Schaltpunkte des Triggers, diese werden durch den von der
Ausgangsspannung beeinflussten Spannungsabfall am Widerstand $R_1$ verschoben. Die aus der Simulation
berechneten Schaltpunkte sind, $U_{low}=0,963V$ und $U_{high}=2,728V$. Diese sind auch mit kleinen
Abweichungen in der Kennlinie zu erkennen, $U_{low}\approx 1,1V$ und $U_{high}\approx 2,25V$.\\
\\
Der Schmitt-Trigger wird zur digitalisierung von Signalen eingesetzt. Eine Hysterese wird hierbei benötigt,
da an den Schaltflanken oft Störsignale ein mehrfaches Über- bzw. Untersteuern des Operationsverstärkers verursachen, was
zu falschen Codierungen führen kann.

% !TEX root = deckblatt3b.tex

\section{Anhang}

\begin{figure}[H]
\centering
\resizebox{0.5\linewidth}{!}{
  \csvautobooktabular{./csv_files/bode1_messdaten.csv}
  }
  \caption{Messwerte f\"ur Figure 11 \& 12}
\end{figure}

\begin{figure}[H]
  \centering
  \resizebox{0.5\linewidth}{!}{
  \csvautobooktabular{./csv_files/bode2_messdaten.csv}
  }
 \caption{Messwerte f\"ur Figure 13 \& 14}
\end{figure}


\end{document}
